\chapter{Stratégie de Données et Prétraitement}

\section{Sourcing et acquisition des bases de données}

\subsection{Piste A : Collaboration avec la Direction IARD} Exploitation des données réelles de l'assureur et sollicitation du réseau du directeur pour obtenir des bases qualifiées.

\subsection{Piste B : Recherche de bases de données externes (Kaggle)} Utilisation de jeux de données issus de compétitions de data science spécialisées en détection d'anomalies.

\subsection{Piste C : Open Data et génération de données synthétiques} Recours à des bases anonymisées disponibles en ligne ou création de données via des techniques de falsification (expérience du TER).

\section{Préparation et enrichissement des données}

\subsection{Traitement des données structurées} Nettoyage des bases de sinistres classiques (âge du conducteur, lieu, montant, type de garantie).

\subsection{Extraction de caractéristiques non-structurées} Utilisation du NLP pour les rapports d'experts et des Transformers pour l'analyse des pièces justificatives.

\section{Gestion du déséquilibre des classes et falsification}

\subsection{La problématique de la rareté des cas de fraude} Analyse statistique du faible taux de fraude dans un portefeuille sain.

\subsection{Techniques de ré-échantillonnage et GANs} Utilisation de réseaux antagonistes génératifs (GANs) pour créer des exemples de fraude réalistes à partir de ton projet sur les faux documents (CNI, factures).