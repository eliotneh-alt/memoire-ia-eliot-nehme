\chapter{Contexte réglementaire et modélisation en assurance vie}
\label{chap:contexte}
\newpage
\section{Les spécificités des produits d'assurance vie épargne}
\label{sec:spec_av}

\subsection{Principes fondamentaux du contrat d'assurance vie}

L'assurance vie est une convention par laquelle un assureur, en contrepartie du versement de primes, s'engage à verser un capital ou une rente à la survenance d'un événement incertain lié à la durée de la vie humaine. Cet événement, qui constitue l'aléa au cœur du contrat, peut être le décès de l'assuré avant une date donnée ou, à l'inverse, sa survie jusqu'à cette date. Ce mécanisme repose sur un cycle de production inversé : l'assureur perçoit les primes bien avant de devoir potentiellement régler les prestations, ce qui l'amène à investir ces sommes sur des horizons de temps longs pour honorer ses engagements futurs.

\begin{figure}[H]
    \centering
    \includegraphics[width=0.8\textwidth]{images/2_chapitres/chapitre1/cycle-de-production.jpg}
    \caption{Cycle de production inversé en assurance vie \textbf{(graphique temporaire)}}
    \label{fig:cycle_production_inverse}
\end{figure}

La nature de ces engagements répond à des objectifs variés. Les contrats en cas de vie prévoient le versement d'un capital ou d'une rente à une échéance prévue si l'assuré est en vie ; ils sont typiquement utilisés pour se constituer un complément de retraite ou une épargne de précaution. À l'opposé, les contrats en cas de décès garantissent le versement d'un capital ou d'une rente au(x) bénéficiaire(s) désigné(s) au décès de l'assuré, souvent pour protéger des proches ou anticiper des droits de succession. Il existe également des contrats mixtes qui combinent ces deux garanties.


Le fonctionnement de ces contrats repose sur la capitalisation : les primes versées sont investies pour financer la propre couverture future de l'assuré. De par leur nature, ces engagements s'étendent sur de très longues périodes. Une caractéristique fondamentale de l'assurance vie française est sa liquidité. L'assuré dispose de la possibilité de récupérer son épargne à tout moment via un rachat, qui peut être partiel ou total. Cette faculté de rachat constitue une option dont la valeur et le risque doivent être finement gérés par l'assureur, car son exercice a un impact direct sur les besoins de liquidité du portefeuille. La fiscalité joue un rôle incitatif majeur car les plus-values sont imposées plus lourdement si le rachat intervient avant la huitième année du contrat. Ceci encourage alors l'épargne de long terme.

La gestion de ces engagements de long terme amène l'assureur à proposer différentes modalités d'investissement. Celles-ci permettent de répartir le risque financier entre l'assuré et l'assureur, définissant ainsi le profil de rendement potentiel du contrat. Un contrat plus sûr aura des possibilités de rendements plus faible qu'un contrat risqué. L'épargne des assurés peut ainsi être investie sur deux principaux types de supports aux profils de risque distincts, qui peuvent être combinés au sein de différents types de contrats.