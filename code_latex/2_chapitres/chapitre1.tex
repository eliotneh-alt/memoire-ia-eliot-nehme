\chapter{Enjeux économiques de la fraude}
\label{chap:contexte}
\newpage
\section{La fraude en assurance non-vie (auto/MRH)}
\label{sec:spec_av}

\subsection{Définitions, typologies (fraude opportuniste vs organisée).}
\subsubsection{La fraude à la souscription}
\subsubsection{La fraude lors de la survenance d'un sinistre}
\subsection{La fraude liée aux conditions générales}

\section{Les sanctions et le cadre réglementaire}
\section{Le dilemme de le rentabilité }

\subsection{Analyse du coût de l'investigation (experts, avocats) par rapport au montant de l'enjeu. Le seuil de déclenchement d'une enquête.}

\section{Impact sur la tarification}

\subsection{Biais dans les statistiques de sinistralité (fréquence et coût moyen).}
\subsection{Conséquence sur la prime pure : comment la fraude fait payer les assurés honnêtes.}

