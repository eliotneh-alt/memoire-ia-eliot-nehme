\chapter{Enjeux économiques et actuariels de la fraude}
\label{chap:contexte}
\newpage
\section{La fraude en assurance non-vie (auto/MRH)}
\label{sec:spec_av}

\subsection{Définitions, typologies (fraude opportuniste vs organisée).}

La fraude en assurance non-vie est un act intentionnel visant à tromper une compagnie d'assurance dans le but d'obtenir un avantage financier, de la part d'une personne morale ou physique.

\subsubsection{La fraude à la souscription}

L'assurance peut être victime de fraude dès la phase de souscription du contrat. Cela peut inclure la fourniture d'informations fausses ou incomplètes sur le demandeur d'assurance, telles que l'omission de déclarer des antécédents médicaux, des informations sur le véhicule ou le lieu de résidence. Ces fraudes visent à obtenir des primes plus basses ou à garantir la couverture d'un risque qui serait autrement exclu.


\subsubsection{La fraude lors de la survenance d'un sinistre}

La fraude la plus courante en assurance non-vie se produit lors de la déclaration d'un sinistre. Cela peut inclure la surestimation des dommages, la déclaration de sinistres fictifs, la falsification de preuves (comme des photos ou des rapports de police), ou tout simplement inventer des sinistres artificiels. Ces fraudes visent à obtenir des indemnités plus élevées que celles auxquelles l'assuré a droit, ou bien créer une indémnité qui n'aurait tout simplement pas du exister.

\subsubsection{La fraude liée aux conditions générales}

La fraude peut également se produire en exploitant les conditions générales des contrats d'assurance. Par exemple, un assuré peut tenter de contourner les exclusions de couverture en manipulant les circonstances d'un sinistre ou en exploitant des ambiguïtés dans le contrat. La fraude documentaire (faux certificats, fausses factures, photos retouchées) est en explosion et représente désormais 39\% (ALFA) des cas de fraude détectés en IARD. Cela prouve que les méthodes classiques ne suffisent plus et qu'il faut analyser les documents eux-mêmes.

\section{Les sanctions et le cadre réglementaire}

\subsection{Souscription}

Le code des assurances article L113-8 stipule que toute fausse déclaration intentionnelle de la part de l'assuré lors de la souscription d'un contrat d'assurance entraine la nullité du contrat. En cas de découverte de la fraude, l'assureur peut résilier le contrat et refuser toute indemnisation en cas de sinistre. De plus, les primes acquittées restent acquises à l'assureur en guise de dommages et intérêts. 

\subsection{Sinistre}

Selon l'article L113-9 du code des assurances, en cas de fraude lors de la déclaration d'un sinistre, l'assureur peut refuser d'indemniser l'assuré. De plus, si la fraude est découverte après le paiement de l'indemnité, l'assureur peut exiger le remboursement des sommes versées. Des sanctions pénales peuvent également être appliquées, incluant des amendes et des peines d'emprisonnement, en fonction de la gravité de la fraude.

\section{Le dilemme de le rentabilité}

\subsection{L'ampleur du marché de la fraude}

Le marché de l'assurance est si important en France qu'il n'est pas possible de tout vérifier à la main. En 2023, les assureurs ont versé 56,4 milliards d'euros de prestations (sinistres et indemnités). La fraude représente environ 10\% de ce montant, soit 5,6 milliards d'euros. Cela signifie que les assureurs doivent investir massivement dans les contrôles et les investigations pour limiter ces pertes.

\subsection{Analyse du coût de l'investigation (experts, avocats) par rapport au montant de l'enjeu. Le seuil de déclenchement d'une enquête}

\section{Impact sur la tarification}

\subsection{Biais dans les statistiques de sinistralité (fréquence et coût moyen).}
\subsection{Conséquence sur la prime pure : comment la fraude fait payer les assurés honnêtes.}

