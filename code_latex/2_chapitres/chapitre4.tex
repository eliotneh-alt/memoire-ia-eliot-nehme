\chapter{Mise en œuvre technique et détection par l'IA}

\section{Détection de la fraude opportuniste (Auto et MRH)}

\subsection{L'IA au service de l'expertise visuelle (YOLO)} Remplacer l'œil de l'expert pour détecter les incohérences : identifier si une griffure latérale est compatible avec un choc frontal sur un poteau.

\subsection{Analyse de la falsification documentaire} Détection de retouches sur les photos d'accidents ou sur les documents d'identité.

\section{Détection de la fraude en bande organisée}

\subsection{Identification de patterns de coûts identiques} Repérer les sinistres ayant exactement le même coût moyen, signe potentiel d'une fraude industrielle.

\subsection{Analyse spatiale et comportementale} Détection de sinistres similaires sur des habitations éloignées (MRH) n'ayant aucun lien logique, mais présentant des caractéristiques techniques identiques.

\section{Évaluation de la performance des modèles}

\subsection{Métriques de précision et de rappel} Mesurer la capacité du modèle à ne pas oublier de fraudeurs tout en évitant de suspecter des clients honnêtes.

\subsection{Interprétabilité des résultats (XAI)} Expliquer pourquoi l'IA a flagué un dossier pour que l'enquêteur humain puisse prendre le relais.